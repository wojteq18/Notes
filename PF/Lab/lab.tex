\documentclass[11pt,a4paper]{article}

% Kodowanie i obsługa języka polskiego
\usepackage[utf8]{inputenc}      % Kodowanie wejścia
\usepackage[T1]{fontenc}         % Kodowanie fontów
\usepackage[polish]{babel}       % Obsługa języka polskiego

% Pakiety matematyczne i inne przydatne
\usepackage{amsmath, amssymb, amsthm}  % Pakiety do matematyki
\usepackage{graphicx}                  % Obsługa grafiki
\usepackage{hyperref}                  % Linki i spis treści
\usepackage{geometry}                  % Ustawienie marginesów
\usepackage{algorithm}               % Algorytmy
\usepackage{algpseudocode}           % Pseudokod
\usepackage{fancyvrb}                  % boxy wokol sekcji
\usepackage{listings}               



\title{Programowanie funkcyjne - laboratoria}
\author{Wojciech Typer}
\date{}

\begin{document}
\maketitle
\vspace{1\baselineskip}
\textbf{zadanie 1}
\vspace{1\baselineskip}
power x y = power $y^x$ \par
p2 = power 4 $\rightarrow$ power 4 y = $y^4$ \par
p3 = power 3 \par
(p2 . p3) 2 = p2(p32) = p 2 8 = $8^4$ = 4096 \par
p2 :: Int -> Int \par
p3 :: Int -> Int \par
(p2 . p3) :: Int -> Int \par
Wyrażenia lambda: \par
power = $/x \rightarrow /y \rightarrow y ^ x$ \par
p2 = $/y \rightarrow y ^ 4$ \par
p3 = $/y \rightarrow y ^ 3$ \par
\vspace{1\baselineskip}
\textbf{zadanie 4} \par
\vspace{1\baselineskip}
plus = $\lambda$ xy -> x + y \par
multi = $\lambda$ xy -> x * y \par
\vspace{1\baselineskip}
\textbf{zadanie 5} \par
\vspace{1\baselineskip}
haskell: \par
$\lambda x \rightarrow 1 + x * (x + 1)$ \par
python: \par
f = lambda x: 1 + x * (x + 1) \par
\vspace{1\baselineskip}
\textbf{zadanie 6} \par
\vspace{1\baselineskip}
Ustalmy zbiory \( A, B, C \). Niech

\[
\text{curry} : C^{B \times A} \to (C^B)^A
\]

będzie funkcją zadaną wzorem:

\[
\text{curry}(\varphi) = \lambda a \in A \to (\lambda b \in B \to \varphi(b, a)).
\]

oraz niech

\[
\text{uncurry} : (C^B)^A \to C^{B \times A}
\]

będzie zadana wzorem:

\[
\text{uncurry}(\psi)(b, a) = (\psi(a))(b).
\]

\begin{enumerate}
    \item Pokaż, że \( \text{curry} \circ \text{uncurry} = \text{id}_{(C^B)^A} \) oraz \( \text{uncurry} \circ \text{curry} = \text{id}_{C^{B \times A}} \).

    \item Wywnioskuj z tego, że \( |(C^B)^A| = |C^{B \times A}| \). Przypomnij sobie dowód tego twierdzenia, który poznałeś na pierwszym semestrze studiów.

    \item Spróbuj zdefiniować w języku Haskell odpowiedniki funkcji \texttt{curry} i \texttt{uncurry}.
\end{enumerate}

\bigskip
\hrule
\bigskip

\begin{enumerate}
    \item Pokażemy, że \( \text{curry} \circ \text{uncurry} = \text{id}_{(C^B)^A} \) oraz \( \text{uncurry} \circ \text{curry} = \text{id}_{C^{B \times A}} \).
        \begin{itemize}
            \item \( \text{curry} \circ \text{uncurry} \)
                \begin{equation}
                    \begin{aligned}
                        (\text{curry} \circ \text{uncurry})(\psi)
                        &= \text{curry}(\text{uncurry}(\psi)) \\
                        &= \text{curry}(\lambda a \in A \to (\lambda b \in B \to \psi(a)(b))) \\
                        &= \lambda a \in A \to (\lambda b \in B \to \psi(a)(b)).
                    \end{aligned}
                \end{equation}

            \item \( \text{uncurry} \circ \text{curry} \)
                \begin{equation}
                    \begin{aligned}
                        (\text{uncurry} \circ \text{curry})(\varphi)
                        &= \text{uncurry}(\text{curry}(\varphi)) \\
                        &= \text{uncurry}(\lambda a \in A \to (\lambda b \in B \to \varphi(b, a))) \\
                        &= \lambda b \in B \to (\lambda a \in A \to \varphi(b, a)).
                    \end{aligned}
                \end{equation}
        \end{itemize}
        Z powyższych równań wynika, że \( \text{curry} \circ \text{uncurry} = \text{id}_{(C^B)^A} \) oraz \( \text{uncurry} \circ \text{curry} = \text{id}_{C^{B \times A}} \). \qed
    \item Możemy pokazać że \texttt{curry} i \texttt{uncurry} są iniekcjami niewprost, nakładając odpowiednio przeciwne funkcje na obie strony równości:
                \begin{itemize}
            \item Załóżmy, że \( \text{curry}(\varphi_1) = \text{curry}(\varphi_2) \). Wtedy:
                \begin{equation}
                    \begin{aligned}
                        \text{curry}(\varphi_1)(a)(b) &= \text{curry}(\varphi_2)(a)(b) \\
                        \varphi_1(b, a) &= \varphi_2(b, a) \\
                        \varphi_1 &= \varphi_2.
                    \end{aligned}
                \end{equation}
            \item Załóżmy, że \( \text{uncurry}(\psi_1) = \text{uncurry}(\psi_2) \). Wtedy:
                \begin{equation}
                    \begin{aligned}
                        \text{uncurry}(\psi_1)(b, a) &= \text{uncurry}(\psi_2)(b, a) \\
                        \psi_1(a)(b) &= \psi_2(a)(b) \\
                        \psi_1 &= \psi_2.
                    \end{aligned}
                \end{equation}
        \end{itemize}
        A więc istnieje biekcja między \( (C^B)^A \) i \( C^{B \times A} \), co oznacza, że te zbiory mają taką samą moc. \qed
    \item W języku Haskell funkcje \texttt{curry} i \texttt{uncurry} można zdefiniować następująco:
        \begin{Verbatim}[frame=single]
curry :: ((b, a) -> c) -> a -> b -> c
curry f x y = f (y, x)
        \end{Verbatim}
        \begin{Verbatim}[frame=single]
uncurry :: (a -> b -> c) -> (a, b) -> c
uncurry f (x, y) = f x y
        \end{Verbatim}

\end{enumerate}

\end{document}