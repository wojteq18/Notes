\documentclass{article}
\usepackage[utf8]{inputenc}
\usepackage[T1]{fontenc}
\usepackage{lmodern}
\usepackage[polish]{babel}
\usepackage{amsmath}
\usepackage{tikz}
\usepackage{algorithm}
\usepackage{algpseudocode}
\usepackage{hyperref}
\usepackage{float}
\usepackage{graphicx}

\title{Wstęp do sztucznej inteligencji}
\author{Wojciech Typer}
\date{}

\begin{document}
\maketitle
\vspace{1\baselineskip}
\textbf{Przestrzeń stanów}  \par
\vspace{1\baselineskip}
Przestrzeń stanów to uporządkowana czwórka (V, E, S, F), gdzie: \par
\begin{itemize}
    \item V $\rightarrow$ zbiór wierzchołków reprezentujących stany powstałe w trakcie rozwiązywania problemów
    \item E $\rightarrow$ zbiór krawędzi reprezentujących możliwe przejścia między stanami
    \item S $\rightarrow$ niepusty podzbiór V, zawierający stany początkowe problemu
    \item F $\rightarrow$ niepusty podzbiór V, zawierający stany docelowe problemu (mogą być zdefiniowane wprost lub przez własności które chcemy osiągnąć)
    \item Rozwiązaniem będziemy nazywać ścieżkę w grafie od stanu początkowego do stanu docelowego
\end{itemize}
\vspace{1\baselineskip}
\textbf{Podstawowe problemy przy przeszukiwaniu przestrzeni stanów}  \par
\vspace{1\baselineskip}
\begin{itemize}
    \item Czy algorytm gwarantuje znalezienie rozwiązania?
    \item Czy algorytm zawsze się kończy?
    \item Czy algorytm znajduje optymalne rozwiązanie?
    \item Jaka jest złożoność czasowa i pamięciowa algorytmu?
    \item Jak można poprawić złożoność czasową i pamięciową?
\end{itemize}
\vspace{1\baselineskip}
\begin{figure}[H]
    \centering
    \includegraphics[width=0.5\textwidth]{/home/wojteq18/Pobrane/zdjecia/stany.jpg}
    \label{fig:example_image}
\end{figure}
\vspace{1\baselineskip}
\textbf{Sformułowanie zadania dla algorytmów przeszukiwania}  \par
\vspace{1\baselineskip}
\begin{itemize}
    \item Precyzyjna definicja przestrzeni stanów
    \item określenie stanu początkowego
    \item określenie reguł przejścia między stanami (operatory akcji lub funkcja następnika)
    \item zbiór stanów docelowych lub funkcja weryfikacji osiągnięcia celu
    \item funkcja kosztu ścieżki
\end{itemize}
\vspace{1\baselineskip}
\textbf{Kierunki przeszukiwania}  \par
\vspace{1\baselineskip}
\begin{itemize}
    \item w przód (od stanu początkowego do celu)
    \item w tył (od stanu celu do początku)
    \item przeszukiwanie dwukierunkowe
\end{itemize}
\vspace{1\baselineskip}
\begin{figure}[H]
    \centering
    \includegraphics[width=0.7\textwidth]{/home/wojteq18/Pobrane/zdjecia/wsi2.png}
    \label{fig:example_image}
\end{figure}
\vspace{1\baselineskip}

\vspace{1\baselineskip}
\textbf{Strategie przeszukiwania} \par
\vspace{1\baselineskip}
\begin{itemize}
    \item Przeszukiwanie w głąb
        \begin{itemize}
            \item zaczynamy w wierzchołu początkowym
            \item dla aktualnego wierzchołka v:
                \begin{itemize}
                    \item oznacz v jako zbadany
                    \item jeśli v jest celem zakończ procedurę
                    \item jeśli niezbadany jest wierzchołek do którego można przejść to przejdź do niego
                    \item jeśli nie ma już niezbadanych sąsiadów to wróć do wierzchołka z którego przyszedłeś
                \end{itemize}
                Zalety: łatwość implementacji, małe wymagania pamięciowe \par
                Wady: znalezione rozwiązanie nie musi być optymalne
        \end{itemize}
        \begin{figure}[H]
            \centering
            \includegraphics[width=0.7\textwidth]{/home/wojteq18/Pobrane/zdjecia/dfs.jpg}
            \label{fig:dfs_image}
        \end{figure}
    \item Przeszukiwanie w głąb ze stosem
        \begin{itemize}
            \item zaczynamy w wierzchołku początkowym
            \item dla aktualnego wierzchołka v:
                \begin{itemize}
                    \item oznacz v jako zbadany
                    \item jeśli v jest celem zakończ procedurę
                    \item dla każdego sąsiada v, który jest niezbadany, dodaj go na stos i oznacz jako odwiedzony
                    \item jako następny, weź wierzchołek ze szczytu stosu
                \end{itemize}
                Zalety: łatwość implementacji, łatwe struktury pamięciowe \par
                Wady: znalezione rozwiązanie nie musi być optymalne
        \end{itemize}
    \item Przeszukiwanie wszerz
        \begin{itemize}
            \item zaczynamy w wierzchołku początkowym
            \item dla aktualnego wierzchołka v:
                \begin{itemize}
                    \item oznacz v jako zbadany
                    \item jeśli v jest celem zakończ procedurę
                    \item dla każdego sąsiada v, który jest niezbadany, dodaj go do kolejki i oznacz jako odwiedzony
                    \item jako następny, weź wierzchołek z początku kolejki
                \end{itemize}
                Zalety: znalezione rozwiązanie jest optymalne \par
                Wady: duże wymagania pamięciowe
        \end{itemize}
        \begin{figure}[H]
            \centering
            \includegraphics[width=0.7\textwidth]{/home/wojteq18/Pobrane/zdjecia/bs.jpg}
            \label{fig:dfs_image}
        \end{figure}
    \item Best-First-Search
        \begin{itemize}
            \item Rozszerzenie przeszukiwania wszerz po dodaniu funkcji kosztu, która eksploruje graf poprzez rozwinięcie najbardziej obiecującego węzła wybranego zgodnie z określoną regułą.
                \begin{itemize}
                    \item zaczymay w wierzchołku początkowym, którego koszt ustwiamy na 0
                    \item dla akutalnego wierzchołka v:
                        \begin{itemize}
                            \item oznacz v jako zbadany
                            \item jeśli v jest celem zakończ procedurę
                            \item dla każdego sąsiada v który jest odwiedzony, jeśli koszt v plus koszt przejścia są mniejsze niż jego dotychczasowy koszt
                            to zmodyfikuj go w kolejce priorytetowej nadając nowy mniejszy koszt
                            \item dla każdego sąsiada v, który jest niezbadany, dodaj go do kolejki priorytetowej z kosztem v plus koszt przejścia oraz oznacz jako odwiedzony
                            \item jako następny, weź wierzchołek z kolejki priorytetowej o najmniejszym koszcie
                        \end{itemize}
                \end{itemize}
                Zalety: znalezienie rozwiązania jest optymalne, ze względu na koszt ścieżki \par
                Wady: kolejka priorytetowa jest trudniejsza w implementacji i ma większą złożoność czasową
        \end{itemize}
        \begin{figure}[H]
            \centering
            \includegraphics[width=0.7\textwidth]{/home/wojteq18/Pobrane/zdjecia/image-52.png}
            \label{fig:dfs_image}
        \end{figure}
    \item A* Search 

\end{itemize}

\end{document}
