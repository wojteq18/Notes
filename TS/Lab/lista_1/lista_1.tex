\documentclass{article}
\usepackage[utf8]{inputenc}
\usepackage[T1]{fontenc}
\usepackage{lmodern}
\usepackage[polish]{babel}
\usepackage{amsmath}
\usepackage{tikz}
\usepackage{algorithm}
\usepackage{algpseudocode}
\usepackage{hyperref}
\usepackage{float}
\usepackage{graphicx}

\title{Technologie sieciowe - lista 1}
\author{Wojciech Typer}
\date{}

\begin{document}
\maketitle
\vspace{1\baselineskip}
\textbf{Pingowanie serwerów (rozmiar pakietów: 64 bajty)}
\vspace{1\baselineskip}
\begin{itemize}
    \item Serwer w Polsce - Politechnika Wrocławska
    \begin{itemize}
        \item ttl = 52
        \item time $\approx$ 6.05 ms 
        \item hops = $64 - 52 = 12$
    \end{itemize}
    \item Serwer google.com
    \begin{itemize}
        \item ttl = 58
        \item time $\approx$ 11.75 ms
        \item hops = $64 - 58 = 6$
    \end{itemize}
    \item Serwer w Australii - sydney.edu.au
    \begin{itemize}
        \item ttl = 105
        \item time $\approx$ 319 ms
        \item hops = $128 - 105 = 23$
    \end{itemize}
    \item Serwer w Czechach - cuni.cz 
    \begin{itemize}
        \item ttl = 53
        \item time $\approx$ 19.1 ms
        \item hops = $64 - 53 = 11$
    \end{itemize} 
    \item Serwer w Chinach - fudan.edu.cn 
    \begin{itemize}
        \item ttl = 221
        \item time $\approx$ 416.75 ms
        \item hops = $256 - 221 = 35$
    \end{itemize}
    \item Serwer w Japonii - www.kyoto-u.ac.jp
    \begin{itemize}
        \item ttl = 52
        \item time $\approx$ 34.7 ms
        \item hops = $64 - 52 = 12$
    \end{itemize}
    \item Serwer w Niemczech - www.hu-berlin.de
    \begin{itemize}
        \item ttl = 48
        \item time $\approx$ 43.7 ms
        \item hops = $64 - 48 = 16$
    \end{itemize}
\end{itemize}
\vspace{1\baselineskip}
\textbf{Obserwacje:}
\vspace{1\baselineskip}
\begin{itemize}
    \item Liczba przeskoków (hops)
    \begin{itemize}
        \item Najmniejsza liczba przeskoków (hops = 6) wystąpiła w przypadku serwera Google w usa, co może wynikać z wielu serwerów cache'ujących Google w Europie
        \item Największa liczba przeskoków (hops = 35) wystąpiła w serwerach w Chinach, co sugeruje, że pakiet przeszedł przez wiele pośrednich routerów i
        prawdopodobnie przez chińską sieć zaporową ("Great Firewall")
        \item Serwery w sąsiednich krajach (Czechy, Niemcy) mają stosunkowo małą liczbę przeskoków, co jest zgodne z ich bliską geograficzną lokalizacją
    \end{itemize}
    \item Opóźnienia (time) a odegłości geograficzne
    \begin{itemize}
        \item Najkrótsze czas odpowiedzi miał serwer w Polsce (Politechniki Wrocławskiej), co jest zgodne z bliską lokalizacją geograficzną
        \item Najdłuższy czas odpowiedzi miał serwer chiński, co może być skutkiem restrykcji sieciowych w Chinach
        \item Serwer w Australii miał stosunkowo długi czas odpowiedzi, co jest zgodne z dużą odległością geograficzną
    \end{itemize}
\end{itemize}
\vspace{1\baselineskip}
\textbf{Pingowanie serwerów (rozmiar pakietów: 1472 bajtów)}
\vspace{1\baselineskip}
\begin{itemize}

    \item Serwer google.com
    \begin{itemize}
        \item ttl = 58
        \item time $\approx$ 12.35 ms
        \item hops = $64 - 58 = 6$
    \end{itemize}

    \item Serwer w Czechach - cuni.cz 
    \begin{itemize}
        \item ttl = 53
        \item time $\approx$ 21.5 ms
        \item hops = $64 - 53 = 11$
    \end{itemize} 
    \item Serwer w Chinach - fudan.edu.cn 
    \begin{itemize}
        \item ttl = 221
        \item time $\approx$ 416.5 ms
        \item hops = $256 - 221 = 35$
    \end{itemize}
    \item Serwer w Japonii - www.kyoto-u.ac.jp
    \begin{itemize}
        \item ttl = 52
        \item time $\approx$ 36.5 ms
        \item hops = $64 - 52 = 12$
    \end{itemize}
    \item Serwer w Niemczech - www.hu-berlin.de
    \begin{itemize}
        \item ttl = 48
        \item time $\approx$ 53.6 ms
        \item hops = $64 - 48 = 16$
    \end{itemize}
\end{itemize}
1427 bajty to maksymalny rozmiar pakietu, który można przesłać bez fragmentacji w sieci. \par
\vspace{1\baselineskip}
\textbf{Obserwacje: }
\vspace{1\baselineskip}
\begin{itemize}
    \item Problemy z dostarczeniem pakietów do niektórych serwerów: Na serwery Politechniki Wrocławskiej i Uniwerystety w Sydney nie udało
    się przesłać tak dużych pakietów, prawdopodobnie z powodu ograniczeń MTU (Maximum Transmission Unit) lub obecności firewalli
    \item Czas odpowiedzi od serwerów, które otrzymały pakiet wzrósł, co może sugerować, że większe pakiety wymagają dłuższego czasu
    przetworzenia przez routery
    \item Liczba przeskoków (hops) pozostała taka sama, co sugeruje, że jest niezależna od rozmiaru pakietów
\end{itemize}
\vspace{1\baselineskip}
\textbf{Traceroute: } \par
\vspace{1\baselineskip}
Traceroute to program służący do badania trasy pakietów w sieci IP. 
Internet nie zawsze jest symetryczny - pakiety ,,wychodzące" z komputera mogą podązać inną drogą niz pakiety ,,przychodzące". 
W tym celu możemy wykorzsytać program Traceroute i mtr.
\end{document}