\documentclass{article}
\usepackage[utf8]{inputenc}
\usepackage[T1]{fontenc}
\usepackage{lmodern}
\usepackage[polish]{babel}
\usepackage{amsmath}
\usepackage{tikz}
\usepackage{algorithm}
\usepackage{algpseudocode}
\usepackage{hyperref}
\usepackage{float}
\usepackage{graphicx}

\title{Algorytmy i Struktury Danych - ćwiczenia}
\author{Wojciech Typer}
\date{}

\begin{document}
\maketitle
\textbf{zadanie 1/ lista1} \par
Zadanie sprowadza się do znalezenia najmniejszego n, 
takiego, że: \par
$44n^2 < 2^n$ \par
najmniejszym takim n jest $n = 13$ \par
indukcyjnie można pokazać, że dla każdych następnych n \par
nierówność dalej będzie zachowana: \par
zał: $44n^2 < 2^n$ \par
krok indukcyjny: \par
$44(k + 1)^2 < 2^{k + 1}$ \par
$44(k^2 + 2k + 1) < 2^k * 2$ \par
$44k^2 + 88k + 44 < 2^k * 2$ \par
z założenia mamy, że: $44k^2 < 2^k$ \par
więc musimy pokazać, że: $88k + 44 < 2^k (k \geq 13)$ \par
Ten fragment jest już bardzo łatwo udowodnić indukcyjnie.  \par

\vspace{1\baselineskip}
\textbf{zadanie 2/ lista1} \par
znając $f(n) = t$, musimy znależć n \par
przeliczmy jednostki czasu na mikrosekundy: \par
$1s = 10^6 \mu s$, $30min = 1.8 * 10^9 \mu s$ i $1 wiek = 3.1 * 10^15 \mu s$\par
zatem: \par
$log_{10}(n) = 10^6  \rightarrow n = 10^{60},$ \par $ log_{10}(n) = 1.8 * 10^9  \rightarrow n = 10^1.8 * 10^9,$ \par $ log_{10}(n) = 3.1 * 10^{15}  \rightarrow n = 10^3.1 * 10^{15}$ \par
$\sqrt{n} = 10^6  \rightarrow n = 10^{12}, \sqrt{n} = 1.8 * 10^9  \rightarrow n = 3.24 * 10^{18},$ \par $ \sqrt{n} = 3.1 * 10^{15}  \rightarrow n = 9.61 * 10^{30}$ \par
$2^n = 10^6  \rightarrow n = 19, 2^n = 1.8 * 10^9  \rightarrow n = 30.7, 2^n = 3.1 * 10^{15}  \rightarrow n = 51$ \par
$n! = 10^6  \rightarrow n = 9, n! = 1.8 * 10^9  \rightarrow n = 13, n! = 3.1 * 10^{15}  \rightarrow n = 18$ \par

\vspace{1\baselineskip}
\textbf{zadanie 3/ lista1} \par
1. $e^{\pi}   \rightarrow O(1)$ \par
2. $7(log_{10}(n))^7  \rightarrow O((log(n))^7)$ \par
3. $\sqrt{2\pi n}  \rightarrow O{\sqrt{n}}$ \par
4. $13n + 13  \rightarrow O(n)$ \par
5. $44n^2 * log(n)  \rightarrow O(n^2 * log(n))$ \par
6. $10^n  \rightarrow O(10^n)$ \par
7. $33^n  \rightarrow O(33^n)$ \par 
\end{document}